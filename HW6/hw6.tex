\documentclass[11pt]{article}
\usepackage{../EllioStyle}

\title{Homework 6}
\author{Elliott Pryor}
\date{11 Oct 2020}


\begin{document}
\maketitle

\problem{1} Pg 106 Problem 1

Show that compact sets are closed under arbitrary intersections and finite unions. 
(Hint: You need to show the intersection of finite or infinte compact sets is compact and the union of finitely many compact sets is a compact set.

\hrule


\begin{proof}

We first show that the intersection of finite or infinite compact sets is compact. Let $A$ be the union of any number of compact sets $A_i$.
If $A = \emptyset$ then it is trivially compact. 
So we examine the case where $A \neq \emptyset$. We know by Theorem 3.2.3 that the intersection of any number of closed sets is a closed set. Since a compact set is closed we know that the intersection of any number of these is at least closed.
We show that $A$ must be bounded by contradiction. We assume $A$ is unbounded, so we take a sequence of points $x_1, x_2, ...$ in $A$ that is unbounded. 
Then by the construction of $A$ the sequence $x_1, x_2, ...$ must be in each $A_i$. But $A_i$ is compact so it is closed and bounded, so cannot contain an unbounded sequence. A contradiction. 
So the intersection of any number of compact sets is closed and bounded, so by Theorem 3.3.1 it is compact.

Next we show that the union of a finite number of compact sets is compact. Let $A \cup_{i=1}^n A_i$ where $A_i$ is compact. We do this in much the same way as above.
We know that the union of finitely many closed sets is closed. So the union of a finite number of compact sets is at least closed. 
Then we show that $A$ must be bounded. We assume not, we assume $A$ is unbounded. Then there is a sequence of points $x_1, x_2, ...$ in $A$ that is unbounded. Then $\lim = -\infty$ or $\sup = \infty$. 
Thus there must be infinitely many terms such that $x_j < -n$ or $x_j > n$. Since $A$ is the union of a finite number of sets, by the pigeon hole principle one set $A_i$ must contain infinity many of these. Then $A_i$ is not bounded, a contradiction since $A_i$ is compact. So $A$ is bounded. 
Thus the union of a finite number of compact sets is compact.

\end{proof}


\newpage
\problem{2} Pg 107 Problem 4

If $A \subseteq B_1 \cup B_2$ where $B_1$ and $B_2$ are disjoint open sets and $A$ is compact, show that $A \cap B_1$ is compact. 

Is the same true if $B_1$ and $B_2$ not disjoint?

\hrule

\begin{proof}

So we start with $A \cap B_1$ must be bounded since $A$ is bounded. We then show that $A$ is closed. We consider some sequence of points $x_1, x_2, ...$ in $A \cap B_1$. Since $x_1, x_2, ...$ is also in $A$ it must have some finite limit point $x \in A$. By the construction of $A$ $x \in B_1$ or $x \in B_2$. 
We show that $x \notin B_2$ by contradiction. Suppose $x \in B_2$. By the definition of a limit point in a set, $x$ is a limit point of $B_2$ if for every neighborhood of $x$ there exists a point in $B_2$ not equal to x. So any neighborhood of $x \in B_2$ contains infinitely many points. 
But each $x_1, x_2, ....$ is in $B_1$ which is disjoint from $B_2$. So $B_2$ cannot contain infinitely many points of $x_1, x_2,...$ A contradiction, so $x \in A \cap B_1$ so $A \cap B_1$ is closed and bounded. Thus it is compact.

\end{proof}

No the same is not true if $B_1$ and $B_2$ overlap. $B_2$ could contain a limit point of a sequence $x_1, x_2,...$ in $B_1$ whose limit point is not in $B_1$. For example $B_1 = (0,1)$ and $B_2 = (0.75, 2)$. Then if $A = [0.5, 1.5]$ $A$ is certainly compact. But $A \cap B_1$ = $[0.5, 1)$ which is not closed, thus not compact.


\newpage
\problem{3} Pg 107 Problem 8

If $A$ is compact, show that $\sup A$ and $\inf A$ belong to $A$. 

Give an example of a non-compact set $A$ such that both $\sup A$ and $\inf A$ belong to $A$.

\hrule


\begin{proof}

Given a compact set $A$ we show that $\sup A \in A$ and $\inf A \in A$ by contradiction. 
Suppose $\sup A \notin A$ and $\inf A \notin A$. Then there must be a sequence $x_1, x_2, ...$ in $A$ whose limit point is $\sup A$ and a sequence $y_1, y_2, ...$ in $A$ whose limit point is $\inf A$.
If this were not the case, then either $\sup A$ and $\inf A$ are singular points in $A$, thus a contradiction. Or $\sup A$ is not the least upper bound and $\inf A$ is not the greatest lower bound, a contradiction of the definition of $\sup$ and $\inf$.
 Then $A$ does not contain the limit points of any sequence of points in $A$, a contradiction since $A$ is compact. 
\end{proof}

If $A = [0, 1) \cup (1, 2]$ then the $\inf A = 0$ and $\sup A = 2$ both of which are in $A$, but it is not compact since it does not contain the point $1$. Ie. there is a sequence $x_n = 1/n + 1$ in $A$ whose limit point is clearly $1$, but $1 \notin A$ so $A$ is not compact.


\end{document}