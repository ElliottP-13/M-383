\documentclass[11pt]{article}
\usepackage{../EllioStyle}

\title{Final Exam}
\author{Elliott Pryor\\
-02508408}
\date{20 Nov 2020}
\lhead{Final Exam}
\rhead{Elliott Pryor \quad -02508408}


\renewcommand{\sp}{\; \;}

\begin{document}
\maketitle

\large{\textbf{Problem 1}}

\begin{enumerate}[a)]
	\item Let $f$ be a continuous function defined on an open domain, then it is possible that the inverse image of an open set is not an open set.
	
	False. Otherwise $f$ assumes same value twice, and $f^{-1}$ DNE.
	
	\item If $f$ is a continuous function defined on $[0, 1]$, then it is possible that the image of $f$ is unbounded.
	
	False. Continuous function on compact domain Theorem 4.2.3
	
	\item If $f(x)$ is differentiable on open interval $(a,b)$ then its derivative $f'(x)$ can not have any jump discontinuities on $(a,b)$
	
	True
	
	\item $x - \sin(x) = o (x^2)$ as $x \to 0$
	
	True
	
	\item If $f(x)$ is strictly increasing at $x_0$ and $f$ is differentiable at $x_0$ then $f'(x_0) > 0$.
	
	False

\end{enumerate}

\problem{2}

Show that every infinite compact set has a limit-point. Is the same true for infinite closed
set?

\hrule


\begin{proof}

Suppose not, suppose that an infinite compact set $A$ has no limit points. Then any point $x$ is not a limit point, so $\exists 1/n \sp st. \sp \forall y \in A, \sp y \neq x \sp |y - x| \geq 1/n$. 
Let $a = inf A, \sp b = sup A$. We know $a,b$ must be finite since $A$ is compact and thus bounded (theorem 3.3.1).
Then, the points of $A$ must be separated by at least $1/n$. For any $x \in A$ we know that $a \leq x \leq b$. So there is at most a finite number of values in $A$, a contradiction. So there must be a limit point.
\end{proof}

The same is not true for an infinite closed set. It could contain no limit points and would thus also be closed.





\problem{3}

If $f$ is a continuous function on $\reals$, is it true that $x$ is a limit-point of $x_1, x_2, ...$ implies $f(x)$ is a limit point of $f(x_1), f(x_2), ...$? Prove your conclusion.

\hrule

Yes

\begin{proof}


By theorem 4.1.2, a function $f$ on domain $\mathbb{D}$ is continuous iff for every sequence of points $x_1, x_2, ...$ that has a limit in $\mathbb{D}$ the sequence $f(x_1), f(x_2), ...$ is convergent. 
Since $x$ is a real number then $x \in \mathbb{D}$, 
and since $f$ is continuous we have that $f(x_1), f(x_2),...$ is convergent to $f(x)$ and thus $f(x)$ is a limit point.



\end{proof}




\problem{4}

Function $f(x)$ is defined by:
$$f(x) = \begin{cases}
x^2 \cos(1/x^2), & x \neq 0\\
0, & x = 0
\end{cases}$$

Show that $f$ is differentiable at all $x \in \reals$. Is $f'(x)$ continuous at $x = 0$? Prove your statement.

\hrule

\begin{proof}

So first we show that $f'$ exists at $x \neq 0$. Let $g(x) = x^2$ and $H(x) = \cos(1/x^2)$By product rule, $f' = g'(x)H(x) + g(x)H'(x)$. 
We use the chain rule to find $H'(x) = - \sin(1/x^2) (-2/x^3)$. 
We combine this to get: $f'(x) = 2x \cos (1/x^2) + 2/x \sin(1/x^2)$.
This is well defined $\forall x \neq 0$. 

Then at $x = 0$ we must show that the derivative exists. We show that $\lim_{x \to 0} \frac{f(x) - f(0)}{x - 0}$ exists. 

$$\lim_{x \to 0} \frac{f(x) - f(0)}{x - 0} = \lim_{x \to 0} \frac{x^2 \cos(1/x^2)}{x} =  \lim_{x \to 0} x \cos(1/x) = 0$$

So $f$ is differentiable everywhere. But $\lim_{x \to 0} f'(x) = \lim_{x \to 0} 2x \cos (1/x^2) + 2/x \sin(1/x^2) = DNE$ so it is not continuous at $x = 0$

\end{proof}	





\problem{5}

Suppose $f(x)$ is continuously differentiable on an interval $(a,b)$. Prove that on any closed subinterval $[c,d]$ of $(a,b)$, the function is uniformly differentiable in the sense that given any $1/m$ there exists $1/n$ (independent of $x_0, x$) such that for all $x, x_0 \in [c,d]$ $|x - x_0| < 1/n$ we have
$$|f(x) - f(x_0) - f'(x_0)(x-x_0)| \leq \frac{1}{m} |x - x_0|$$

\textit{Hint: Use mean value theorem on $f(x) - f(x_0)$ and the fact that $f'(x)$ is continuous function on the compact set $[c,d]$}

\hrule

\begin{proof}
 
By the mean value theorem $\exists x_1 \in (x_0, x)$ $f'(x_1) = \frac{f(x) - f(x_0)}{x - x_0}$. 
We can arrange this to get $|f(x) - f(x_0) - f'(x_1)(x - x_0)| = 0$. 
We then add $|f'(x_1)(x - x_0) - f'(x_0)(x - x_0)|$ resulting in: $|f(x) - f(x_0) - f'(x_0)(x - x_0)| = |f'(x_1)(x - x_0) - f'(x_0)(x - x_0)|$.
Since $f'(x)$ is continuous on a compact domain, by theorem 4.2.5, $f'$ is uniformly continuous. So independent of $x, x_0$ we have $\forall 1/m \sp \exists 1/n \sp st. \sp \forall x_1, x_0 \sp |x - x_0| < 1/n \implies |f'(x_1) - f'(x_0)| < 1/m$. 
If we multiply this last by $|x - x_0|$ we get $|f'(x_1)(x - x_0) - f'(x_0)(x - x_0)| < 1/m |x - x_0|$.

Thus in conclusion we have, independent of $x, x_0$, $\forall 1/m \sp \exists 1/n \sp st. \sp \forall x_1, x_0 \sp |x - x_0| < 1/n$

$$|f(x) - f(x_0) - f'(x_0)(x - x_0)| = |f'(x_1)(x - x_0) - f'(x_0)(x - x_0)| \leq  1/m |x - x_0|$$



\end{proof}



\end{document}