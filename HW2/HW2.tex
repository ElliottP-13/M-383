\documentclass[11pt]{article}
\usepackage{../EllioStyle}

\title{Homework 2}
\author{Elliott Pryor}
\date{4 Sept 2020}


\begin{document}
\maketitle

\problem{1}
Prove that between any two distinct rational numbers there are infinitely many other rational numbers.
\hrule

\begin{proof}
We first start by showing that between any two distinct rational numbers, there is at least one rational number. Let $p, q \in \rationals$ s.t. $p \neq q$. By definition of $p = \frac{a}{b} = \frac{a * d}{bd}$, $q = \frac{c}{d} = \frac{ c * b}{bd}$. Because $p \neq q$ $ad \neq cb$ so $|ad - cb| \geq 1$. Then we write $p = \frac{2a * d}{2bd}$ and $q = \frac{2 c * b}{2bd}$. Now the difference $|2ad - 2 cb| \geq 2$. So there must be an integer, $x$, in between $2ad$ and $2cb$.  So we have a new rational number $r = \frac{x}{2bd}$ which lies between $p$ and $q$. So we have shown that there is at least one rational number in between any two distinct rational numbers.

Next we show that there are infinitely many rational numbers in between two distinct rational numbers. Let $p, q \in \rationals$ s.t. $p \neq q$. By above, there is a rational number $r$ in between $p$ and $q$. We can then always find a new rational number, $r'$, in between $p$ and the previous $r$ value. So there are infinitely many rational numbers in between $p$ and $q$.

\end{proof}

\end{document}