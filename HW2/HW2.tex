\documentclass[11pt]{article}
\usepackage{../EllioStyle}

\title{Homework 2}
\author{Elliott Pryor}
\date{4 Sept 2020}


\begin{document}
\maketitle

\problem{1}
Prove that between any two distinct rational numbers there are infinitely many other rational numbers.
\hrule

\begin{proof}
We first start by showing that between any two distinct rational numbers, there is at least one rational number. Let $p, q \in \rationals$ s.t. $p \neq q$. By definition of $p = \frac{a}{b} = \frac{a * d}{bd}$, $q = \frac{c}{d} = \frac{ c * b}{bd}$. Because $p \neq q$ $ad \neq cb$ so $|ad - cb| \geq 1$. Then we write $p = \frac{2a * d}{2bd}$ and $q = \frac{2 c * b}{2bd}$. Now the difference $|2ad - 2 cb| \geq 2$. So there must be an integer, $x$, in between $2ad$ and $2cb$.  So we have a new rational number $r = \frac{x}{2bd}$ which lies between $p$ and $q$. So we have shown that there is at least one rational number in between any two distinct rational numbers.

Next we show that there are infinitely many rational numbers in between two distinct rational numbers. Let $p, q \in \rationals$ s.t. $p \neq q$. By above, there is a rational number $r$ in between $p$ and $q$. We can then always find a new rational number, $r'$, in between $p$ and the previous $r$ value. So there are infinitely many rational numbers in between $p$ and $q$.

\end{proof}




\newpage
\problem{2}
What kinds of real numbers are representable by Cauchy sequences of integers.
\hrule

Only integers are representable by Cauchy sequences of integers. 

\begin{proof} 

Let $x_n$ be a Cauchy sequence of integers. Then we know that $\forall n \in \naturals \: \exists m \in \naturals \: s.t \: \forall j,k \geq m \; |x_j - x_k| \leq 1/n$. If $n \geq 2$ then $x_j$ must equal $x_k$. Since $x_j = x_k \in \integers$ the sequence must converge to $x_j$ which is an integer.
\end{proof}




\newpage
\problem{3}
Prove that if a Cauchy sequence of rationals is modified by changing a finite number of terms, the result is an equivalent Cauchy sequence.
\hrule


\begin{proof}

Let $x_n$ be a Cauchy sequence of rationals that has limit of $x$. We know that $\forall n \in \naturals \: \exists m \in \naturals \: s.t \: \forall j,k \geq m \; |x_j - x_k| \leq 1/n$. Then $x'_n$ is the resulting Cauchy sequence after changing a finite number of elements in $x_n$. Let $a$ be the largest index of the modified elements. Because a finite number of elements were modified, $a$ must be finite.

We show that $x $ is equivalent to $x'$. So we are attempting to show 

$\forall n \in \naturals \: \exists m \in \naturals \: s.t \: \forall k \geq m \; |x_k - x'_k| \leq 1/n$. 

We know that $\forall  k > a$ $x_k = x'_k$. We select $m = a + 1$. Then 
$\forall n \in \naturals \: \forall k \geq m \; |x_k - x'_k| = 0 \leq 1/n$ as required. So $x$ and $x'$ are equivalent.

\end{proof}



\newpage
\problem{4}
Can a Cauchy sequence of positive rational numbers be equivalent to a Cauchy sequence of negative rational numbers. 
\hrule

Yes. 

\begin{proof}

We know that two Cauchy sequences are equivalent iff they have the same limit. So we show that there is a sequence of positive rational numbers and a sequence of negative rational numbers that have the same limit. We take a sequence of positive rational numbers that converges to zero. For example the sequence $x_i = 1/i^3$ converges to zero and is strictly positive. We can similarly find a Cauchy sequence of negative numbers that converge to zero. For example the sequence $x_j = -1/i^3$ converges to zero and is strictly negative. Thus we have found two sequences of rational numbers with the same limit so they must be equivalent. 

\end{proof}


\newpage
\problem{5}
Show that if $x_1, x_2, ...$ is a Cauchy sequence of rational numbers there exists a positive integer $N$ such that $x_j \leq N \; \forall j$.
\hrule

\begin{proof}


In the book (and in class) it was shown that every Cauchy sequence converges to some $x \in \reals$. Then because the real number system is an ordered field that includes the integers, we can find an integer $N \in \integers$ s.t. $N > x + 1$. 

However, it is possible that some $x_j > N$. We first show that the number of elements $x_j > y$ is finite. Because $x_i$ is a Cauchy sequence we know that $\forall n \in \naturals \: \exists m \in \naturals \: s.t \: \forall j,k \geq m \; |x_j - x_k| \leq 1/n$. We consider the case where $n = 1$ as it has the loosest bound on the sequence. $m$ must be finite, then by the Cauchy criterion every element $x_j$ s.t. $j > m$ must be within 1 of $x$: $|x_j - x| \leq 1$. So any element that is larger than $N$ must be in the first $m$ elements. Since this is finite, we can find the maximum value $x_p$. Then we set $N =  \lceil max(x+1, x_p) \rceil $. Then $x_j \leq N \; \forall j$ 

\end{proof}

\end{document}