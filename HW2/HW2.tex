\documentclass[11pt]{article}
\usepackage{../EllioStyle}

\title{Homework 2}
\author{Elliott Pryor}
\date{4 Sept 2020}


\begin{document}
\maketitle

\problem{1}
Prove that between any two distinct rational numbers there are infinitely many other rational numbers.
\hrule

\begin{proof}
We first start by showing that between any two distinct rational numbers, there is at least one rational number. Let $p, q \in \rationals$ s.t. $p \neq q$. By definition of $p = \frac{a}{b} = \frac{a * d}{bd}$, $q = \frac{c}{d} = \frac{ c * b}{bd}$. Because $p \neq q$ $ad \neq cb$ so $|ad - cb| \geq 1$. Then we write $p = \frac{2a * d}{2bd}$ and $q = \frac{2 c * b}{2bd}$. Now the difference $|2ad - 2 cb| \geq 2$. So there must be an integer, $x$, in between $2ad$ and $2cb$.  So we have a new rational number $r = \frac{x}{2bd}$ which lies between $p$ and $q$. So we have shown that there is at least one rational number in between any two distinct rational numbers.

Next we show that there are infinitely many rational numbers in between two distinct rational numbers. Let $p, q \in \rationals$ s.t. $p \neq q$. By above, there is a rational number $r$ in between $p$ and $q$. We can then always find a new rational number, $r'$, in between $p$ and the previous $r$ value. So there are infinitely many rational numbers in between $p$ and $q$.

\end{proof}

\problem{2}
What kinds of real numbers are representable by Cauchy sequences of integers.
\hrule

Only integers are representable by Cauchy sequences of integers. 

\begin{proof} 

Let $x_n$ be a Cauchy sequence of integers. Then we know that $\forall n \in \naturals \: \exists m \in \naturals \: s.t \: \forall j,k \geq m \; |x_j - x_k| \leq 1/n$. If $n \geq 2$ then $x_j$ must equal $x_k$. Since $x_j = x_k \in \integers$ the sequence must converge to $x_j$ which is an integer.
\end{proof}

\problem{3}
Prove that if a Cauchy sequence of rationals is modified by changing a finite number of terms, the result is an equivalent Cauchy sequence.
\hrule


\begin{proof}

Let $x_n$ be a Cauchy sequence of rationals that has limit of $x$. We know that $\forall n \in \naturals \: \exists m \in \naturals \: s.t \: \forall j,k \geq m \; |x_j - x_k| \leq 1/n$. Then $x'_n$ is the resulting Cauchy sequence after changing a finite number of elements in $x_n$. Let $a$ be the largest index of the modified elements. Because a finite number of elements were modified, $a$ must be finite.

We show that $x $ is equivalent to $x'$. So we are attempting to show 

$\forall n \in \naturals \: \exists m \in \naturals \: s.t \: \forall k \geq m \; |x_k - x'_k| \leq 1/n$. 

We know that $\forall  k > a$ $x_k = x'_k$. We select $m = a + 1$. Then 
$\forall n \in \naturals \: \forall k \geq m \; |x_k - x'_k| = 0 \leq 1/n$ as required. So $x$ and $x'$ are equivalent.

\end{proof}

\end{document}