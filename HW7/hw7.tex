\documentclass[11pt]{article}
\usepackage{../EllioStyle}

\title{Homework 7}
\author{Elliott Pryor}
\date{19 Oct 2020}

\renewcommand{\sp}{\; \;}

\begin{document}
\maketitle

\problem{1} 4.1.5 Problem 2

Let $A$ be the set defined by the equations $f_1(x) = 0$, $f_2(x) = 0$, ... $f_n(x) = 0$. Where $f_1, ... f_n$ are continuous functions defined on the whole line. Show that $A$ is closed. Must $A$ be compact? \textit{Hint: } you can use result of 4.1.5 Problem 1 without proof
\hrule

\begin{proof}
So if $\mathcal{D}_i$ is the set of $x$ satisfying $f_i(x) = 0$ then $A = \cup_{i = 1}^n \mathcal{D}_i$. We know that $\mathcal{D}_i = f^{-1}(0)$. 
From 4.1.5 Problem 1 we know that a function $f$ is continuous iff the inverse image of every closed set is a closed set. Since $f_i$ is continuous, $\mathcal{D}_i$ must be closed. 
Then $A$ is the finite union of a finite number of closed sets which is closed by Theorem 3.2.3. So then $A$ must be closed.  
\end{proof}
 
No $A$ is not necessarily compact. If each $f_1$ is a constant function $f_1(x) = 0 \; \forall x \in \reals$. Then $f_1^{-1} = \reals$ which is closed but not compact.




\newpage
\problem{2} 4.1.5 Problem 4

Give a definition of $\lim_{x \to \infty} f(x) = y$. Show that this is true iff for every sequence $x_1, x_2, ...$ of points in the domain of $f$ such that $\lim_{n \to \infty} x_n = \infty$ we have $\lim_{n\to \infty} f(x_n) = y$. \textit{Hint: } For the proof of the 2nd part of the problem, refer to the proof of Theorem 4.1.1.
\hrule

We define $\lim_{x \to \infty} f(x) = y$ as if $\forall 1/m$ there exists an $n$ such that $\forall x > n$ $|f(x) - y| < 1/m$. 

\begin{proof}

We first prove the forward direction. $P \to Q$

Given that $\lim_{x \to \infty} f(x) = y$ exists. We take any sequence of points $x_1, x_2, ...$ such that $\lim_{n \to \infty} x_n = \infty$. We show that this implies $\lim_{n \to \infty} f(x_n) = y$. By our definition, we know that $\forall 1/m$ there exists an $n$ such that $\forall x > n$ $|f(x) - y| < 1/m$. We also know that $\lim_{n \to \infty} x_n = \infty$. Which means that there are infinitely many terms of $x_n$ satisfying $x_j > n$. So $\lim_{j \to \infty} x_j = \infty \implies \exists k \; s.t. \; \forall j \geq k \; x_j > n \implies |f(x_j) - y| < 1/m \implies \lim_{j \to \infty} f(x_j) = y$

We now prove the reverse direction.

Given that for every sequence $x_1, x_2, ...$ of points in the domain of $f$ such that $\lim_{n \to \infty} x_n = \infty$ we have $\lim_{n\to \infty} f(x_n) = y$. 
We show that $lim_{x \to \infty} = y$. We need to show $\forall 1/m$ there exists an $n$ such that $\forall x > n$ $|f(x) - y| < 1/m$. 
Suppose not, suppose that $\exists 1/m$ st. $\forall n$ $\exists n < z_n \in \mathbb{D}$ st. $|f(z_n) - y| \geq 1/m$. We can construct a sequence of the points $z_n$. By definition, $lim_{n \to \infty} = \infty$. But this sequence of points does not converge to $y$. A contradiction. So for every sequence $x_1, x_2, ...$ of points in the domain of $f$ such that $\lim_{n \to \infty} x_n = \infty$ we have $\lim_{n\to \infty} f(x_n) = y$ $\implies$ $\lim_{x \to \infty} f(x) = y$.

\end{proof}



\newpage
\problem{3} 4.1.5 Problem 7

Give an example of a continuous function with domain $\reals$ such that the inverse image of a compact set is not compact. 

\hrule


Let $f = \sin(x)$. Then $\sin(x) \in [-1, 1]$ is compact and $x \in \reals$, but $\arcsin(y) \in (-\infty, \infty)$ if $y \in [-1, 1]$, which is not compact. 



\newpage
\problem{4} 4.1.5 Problem 10

Show that a function that satisfies a Lipschitz condition is uniformly continuous.
\hrule


\begin{proof}

A function satisfies the Lipschitz condition if $\exists m > 0$ st. $|f(x) - f(x_0) | \leq m |x-x_0| \sp \forall x, x_0 \in \mathcal{D}$. 
Then we need to show that $\forall 1/k \sp \exists 1/n \sp s.t. \sp |f(x) - f(x_0)| < 1/k \sp \forall x, x_0 \in \mathcal{D} \text{ satisfying } |x-x_0| < 1/n$. 
We choose $n = k m$ then $|x - x_0| < 1/km$ $\implies |f(x) - f(x_0)| < m * 1/km = 1/k$. 
Thus we have  $\forall 1/k \sp \exists 1/n \sp s.t. \sp |f(x) - f(x_0)| < 1/k \sp \forall x, x_0 \in \mathcal{D} \text{ satisfying } |x-x_0| < 1/n$

\end{proof}



\end{document}