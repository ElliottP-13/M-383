\documentclass[11pt]{article}
\usepackage{../EllioStyle}

\title{Homework 8}
\author{Elliott Pryor}
\date{28 Oct 2020}
\lhead{Elliott Pryor}

\renewcommand{\sp}{\; \;}

\begin{document}
\maketitle

\problem{1} 5.1.3 Problem 1

Show that $f(x) = O(|x - x_0|^2)$ as $x \to x_0$ implies $f(x) = o(|x - x_0|)$ as $x\to x_0$ but give an example to show the converse is not true.

\hrule

\begin{proof}

We know that by the definition of big O $\exists 1/n, c \sp |x - x_0| < 1/n \implies |f(x)| \leq c |g(x)|$. So we have $\exists 1/n, c \sp |x - x_0| < 1/n \implies |f(x)| \leq c |x - x_0|^2$. 
We want to show $\forall 1/m \sp \exists 1/n \sp st \sp |x - x_0| < 1/n \implies |f(x)| < 1/m |x - x_0|$ or equivalently $\lim_{x \to x_0} \frac{f(x)}{|x - x_0|} = 0$. 
Then since $|f(x)| \leq c |x - x_0|^2$ we have $\frac{|f(x)|}{|x - x_0|} \leq c |x - x_0|$ within $x \in (x_0 - 1/n, x_0 + 1/n)$. Then we take the limit and non-strict inequality is preserved so $\lim_{x \to x_0} \frac{|f(x)|}{|x - x_0|} \leq \lim_{x \to x_0} c |x - x_0| = 0$. Since $\frac{|f(x)|}{|x - x_0|} \geq 0 \sp \forall x$ then $\lim_{x \to x_0} \frac{|f(x)|}{|x - x_0|} = 0$ as required.
\end{proof}

For example: if we take the function $f(x) = |x - x_0|^{1.5}$ we have $\lim_{x \to x_0} \frac{|x - x_0|^{1.5}}{|x - x_0|} = \lim_{x \to x_0} \sqrt{|x - x_0|} = 0$ so $f \in o(|x - x_0|)$. 
Then we show that it is not $O(|x-x_0|^2)$ by showing $\frac{|x - x_0|^{1.5}}{|x - x_0|^{2}}$ is unbounded.
 We take $\lim_{x \to x_0} \frac{|x - x_0|^{1.5}}{|x - x_0|^{2}} = \frac{1}{\sqrt{|x-x_0|}} = + \infty$. So there is no constant $c$ that could satisfy $|f(x)| \leq c |x - x_0|^2$


\problem{2} 5.2.4 Problem 1
Let $f$ and $g$ be continuous functions on $[a,b]$ and differentiable at every point in the interior, with $g(a) \neq g(b)$. Prove that there exists a point in $x_0$ in $(a,b)$ such that

$$\frac{f(b) - f(a)}{g(b)-g(a)} = \frac{f'(x_0)}{g'(x_0)}$$

This is also called second mean value theorem

\hrule

\begin{proof}

We let $h(x) = (f(b) - f(a))g(x) - (g(b) - g(a))f(x)$.
In order to apply mean value theorem we need to know $h(b) - h(a)$
\begin{align*}
h(b) - h(a) &= (f(b) - f(a))g(b) - (g(b) - g(a))f(b) - (f(b) - f(a))g(a) + (g(b) - g(a))f(a)\\
&= f(b)g(b) - f(a)g(b) - f(b)g(b) + f(b)g(a) - f(b)g(a) + f(a)g(a) + f(a)g(b) - f(a)g(a)\\
&= 0
\end{align*}


So there is some $x_0 \in (a,b)$ such that $h'(x_0) = 0$. We compute $h'(x) = (f(b) - f(a))g'(x) - (g(b) - g(a))f'(x)$.
So:
\begin{align*}
0 &= (f(b) - f(a))g'(x_0) - (g(b) - g(a))f'(x_0)\\
(g(b) - g(a))f'(x_0) &= (f(b) - f(a))g'(x_0)\\
\frac{f'(x_0)}{g'(x_0)} &= \frac{(f(b) - f(a))}{(g(b) - g(a))}
\end{align*}

\end{proof}



\problem{3} 5.2.4 Problem 2

if $f$ is a function satisfying $f(x) - f(y) \leq M|x - y|^\alpha$
for all $x,y$ and some fixed $M$ and $\alpha > 1$, prove that $f$ is constant.
\textit{Hint: what is }$f'$. It is rumored that a graduate student once wrote a whole
thesis on the class of functions satisfying this condition!

\hrule



We re-write this as $\frac{f(x) - f(y)}{|x-y|} \leq M|x - y|^{\alpha -1}$.
We know that $\alpha > 1$ so $\alpha - 1 > 0$.
We then examine the limit as $x \to y$. 
$$\lim_{x \to y} \frac{f(x) - f(y)}{|x-y|} \leq \lim_{x \to y} M|x - y|^{\alpha -1} = 0$$
We note that this is the definition of the derivative of $f$ at $y$. We have $f'(y) = 0$ at an arbitrary $y$ in the domain, so this could be repeated at every point in the domain and we have $f'(y) = 0 \sp \forall y$. Then the derivative is zero at every point in the domain, so by theorem 5.2.2 $f$ is constant.



\problem{4} 5.2.4 problem 3

Is the converse of the mean value theorem true, in the sense that if $f$ is continuous on $[a,b]$ and differentiable on $(a,b)$, a given point $x_0$ in $(a,b)$ there must exist points $x_1, x_2 \in (a,b)$ such that:

$$\frac{f(x_2) - f(x_1)}{x_2 - x_1} = f'(x_0)$$

\hrule


No

Take $f(x) = x^3$ (or any odd polynomial with repeated root at 0). Choose $x_0 = 0$ then $f'(x_0) = 0$. For any $x_2 > x_1$ in $(a,b)$ we have $\frac{f(x_2) - f(x_1)}{x_2 - x_1} > 0$ since $f$ is monotone increasing. So you cannot find a pair of points $x_1, x_2$ such that $\frac{f(x_2) - f(x_1)}{x_2 - x_1} = f'(x_0)$. So we found a counterexample showing the converse of the mean value theorem cannot be true.



\end{document}