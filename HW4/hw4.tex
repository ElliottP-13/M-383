\documentclass[11pt]{article}
\usepackage{../EllioStyle}

\title{Homework 3}
\author{Elliott Pryor}
\date{16 Sept 2020}


\begin{document}
\maketitle



\problem{1} 2.2.4-3

If $x$ is a real number, show that there exists a Cauchy sequence of rationals, $x_1, x_2, ...$ representing $x$ such that $x_n < x$ for all $n$
\hrule

\begin{proof}


We first show that there is some rational number $y$ s.t $x-1/n \leq y \leq x$ for $x \in \reals$ and $n \in \naturals$. Since the reals are closed under addition, $x - 1/n$ is a real number. By the density of rationals we can find rational numbers $y_1$ and $y_2$ such that $|x - y_1| \leq 1/4n$ and $|(x-1/n) - y_2| \leq 1/4n$. We then let $y$ be the midpoint of $[y_2, y_1]$.  So in the worst case, where $y_1 = x - 1/4n$ and $y_2 = x - 5/4n$ then $y = \frac{(x - 1/4n) - (x - 5/4n)}{2} + (x - 5/4n) = 1/2n + x - 5/4n = x - 3/4n$. 
We then show that $y < x$ by examining the case where $y_1 = x + 1/4n$ and $y_2 = x - 3/4n$ so $y = \frac{x + 1/4n - x + 3/4n}{2} +  x - 3/4n = x - 1/4n < x$.
So $x - y < 1/n$.

From above, we can find some $y \in \rationals$ s.t. $x - y < 1/n$. We then construct sequence of rationals $\{y_k\}$ that satisfy this relation. By the construction, $y_k < x$ $\forall k$. Then $\forall n \in \naturals \; \exists m \in \naturals \; s.t. \; |x - y_k| \leq 1/n \; \forall k \geq m$. By our construction, if $m = n$ the previous statement is true. Therefore, $\{y_k\}$ converges to $x$. Since $\{y_k\}$ is convergent, then it must be Cauchy and it represents $x$ since it has $x$ as its limit.
\end{proof}




\newpage
\problem{2} 2.2.4-7

Prove $|x-y| \geq |x| - |y|$ for any real numbers $x$ and $y$.
\hrule


\begin{proof}

Let $\{x_k\}$ be a Cauchy sequence of rationals representing $x$ and $\{y_k\}$ be a Cauchy sequence of rationals representing $y$. Then $\{x_k - y_k\}$ is a Cauchy sequence representing $x - y$. By the triangle inequality $|x_k - y_k| \geq |x_k| - |y_k|$. By definition $\lim_{k \to \infty} |x_k| = |x|$ and $\lim_{k \to \infty} |y_k| = |y|$. So $lim_{k \to \infty} |x_k - y_k| \geq \lim_{k \to \infty} |x_k| - \lim_{k \to \infty} |y_k| = |x| - |y|$. So we have $|x - y| \geq |x| - |y|$ for some $x,y \in \reals$

\end{proof}



\newpage
\problem{3} 2.3.3-1

Write out a proof that $\lim_{k \to \infty} (x_k + y_k) = x + y$ if $\lim_{k \to \infty} x_k = x$ and $\lim_{k \to \infty} y_k = y$ for sequences of real numbers.
\hrule

\begin{proof}

We know that the sequence $\{x_k\}$ converges to $x$ and $\{y_k\}$ converges to $y$. So $\forall n \in \naturals \; \exists m_1 \in \naturals \; s.t. \; \forall k \geq m_1 \; |x_k - x| \leq 1/2n$ and $\forall n \in \naturals \; \exists m_2 \in \naturals \; s.t. \; \forall k \geq m_2 \; |y_k - y| \leq 1/2n$. Since both $\{x_k\}$ and $\{y_k\}$ have limits, both must be Cauchy sequences. 

So we want to show that $\{x_k + y_k\}$ converges to $x + y$. So we need to show $\forall n \in \naturals \; \exists m \in \naturals \; s.t. \; \forall k \geq m \; |(x + y) - (x_k - y_k)| \leq 1/n$.  We choose $m = \max(m_1, m_2)$ then the following is true.

$$|(x + y) - (x_k + y_k)| = |(x - x_k) + (y - y_k)| \leq |x - x_k| + |y - y_k| \leq 1/2n + 1/2n = 1/n$$
So then by the definition of a limit $\lim_{k \to \infty} (x_k + y_k) = x + y$. 

\end{proof}

\newpage
\problem{4} 2.3.3-3

Let $x_1, x_2, ...$ be a sequence of real numbers such that $|x_n| \leq 1/2^n$, and set $y_n = x_1 + x_2 + ... + x_n$. Show that the sequence $y_1, y_2, ...$ converges. 
\hrule

\begin{proof}

We know that a sequence converges iff it is Cauchy. So we show that $y_1, y_2, ...$ is Cauchy. So we must show $\forall n \in \naturals \; \exists m \in \naturals \; s.t. \; \forall j,k \geq m \; |y_j - y_k| \leq 1/n$. Suppose $j \geq k$ then 

$$|y_j - y_k| = \left|\sum_{i = 1} ^j (1/2)^i - \sum_{i = 1} ^k (1/2)^i \right| = \sum_{i = k} ^j (1/2)^i$$.

 Now we must find an $m$ such that $\sum_{i = m} ^ \infty (1/2)^i \leq 1/n$. If this holds, then $\sum_{i = k} ^j (1/2)^i \leq 1/n$ must also be true since $\sum_{i = k} ^j (1/2)^i < \sum_{i = m} ^\infty (1/2)^i$

Let $s = \sum_{i = 1} ^n (1/2)^i$. Then $2s = 1 + \sum_{i = 1} ^{n-1} (1/2)^i = 1 + s - 1/2^n$. So $s = 1 - 1/2^n$. Then as $\lim_{n \to \infty} s = 1$. So $\sum_{i = 1} ^\infty
 (1/2)^i = s = 1$. 
 
 Then $\sum_{i = m} ^ \infty (1/2)^i = (1/2)^m \sum_{i = 1} ^\infty (1/2)^i = (1/2)^m s = (1/2)^m$. So now we just choose an $m$ such that $1/2^m \leq 1/n$. This holds if $m \geq \frac{\ln(1/n)}{\ln(1/2)}$. For simplicity we choose $m = n$ since $n > \frac{\ln(1/n)}{\ln(1/2)}$ $\forall n \in \naturals$.
 
So we have $\forall n \in \naturals \; \forall j,k \geq n \; |y_j - y_k| \leq 1/n$ as required.

\end{proof}




\end{document}