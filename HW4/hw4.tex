\documentclass[11pt]{article}
\usepackage{../EllioStyle}

\title{Homework 4}
\author{Elliott Pryor}
\date{25 Sept 2020}


\begin{document}
\maketitle



\problem{1} 3.1.3 problem 1a)

Compute the sup, inf limsup, liminf and all the limit points of $x_n = 1 + (-1)^n / n$

\hrule


$x_n = 0, 3/2, 2/3, 5/4, 4/5, 5/6, 6/7, 9/8, ...$

Clearly the sup is 3/2 and inf is 0. 
We then show that the sequence is convergent to 1.
We need to show $\forall n \in \naturals \; \exists m \in \naturals \: s.t. \forall j \geq m \; |x_j - 1| \leq 1/n$. 

$$|x_j - 1| = |1 + (-1)^j / j -1| = |(-1)^j/j| = 1/j$$

If we choose $m = n$ then $1/j \leq 1/n \; \forall j \geq n$ as required. Since it is a convergent sequence, by theorem 3.1.5 $limsup = liminf = 1$.



\newpage
\problem{2} 3.1.3 problem 2

\begin{enumerate}
 \item If a bounded sequence is the sum of a monotone increasing and monotone decreasing sequence ($x_n = y_n + z_n$ where $\{y_n\}$ is monotone increasing and $\{z_n\}$ is monotone decreasing) does it follow that the sequence converges? 
 
 \item What if $\{y_n\}$ and $\{z_n\}$ are bounded?
\end{enumerate}

\hrule


\begin{enumerate}
	\item No, the sequence could oscillate.
	
	\begin{proof} By contradiction
	
	Suppose that sequence $x_n = y_n + z_n$ where $\{y_n\}$ is monotone increasing and $\{z_n\}$ is monotone decreasing converges for any $y_n, z_n$, and $x_n$ is bounded. We let 
	$y_n = \begin{cases}
	n, & \text{if n is even}\\
	(n+1), & \text{if n is odd}
	\end{cases}$ and 
	$z_n = \begin{cases}
	-n, & \text{if n is even}\\
	-(n-1), & \text{if n is odd}
	\end{cases}$
	
	So $y_n = 2, 2, 4, 4, 6, ...$ and $z_n = 0, -2, -2, -4, -4,... $. And then 
	$x_n = (2 + 0), (2 -2), (4 -2), (4 - 4), (6-4), ... = 2, 0, 2, 0, 2, ...$. 
	So clearly $x_n$ is bounded and it does not converge since 
	$|x_n - x_{n+1}| = 2 \; 	\forall n\in \naturals$
	
	So $x_n = y_n + z_n$ does not converge for monotone increasing sequence $y_n$ 
	and monotone decreasing sequence $z_n$. 
	A contradiction, so $x_n$ does not converge for every $y_n, z_n$.
	\end{proof}	 
	
	\item Yes $x_n$ converges if $y_n, z_n$ are bounded. 
	Since $y_n$ is bounded and monotone increasing it must have a finite limit equal to the $sup$, 
	and since $z_n$ is bounded and monotone decreasing it must have a finite limit equal to the $inf$. 
	Thus $\lim_{k \to \infty} y_k = y$ and $\lim_{k \to \infty} z_n = z$.
	Then $\lim_{k \to \infty} y_k + z_k = y + z$.
	Since $y, z \in \reals$ the sequence $x_n = y_n + z_n$ is convergent.
		
	
\end{enumerate}


	


\newpage
\problem{3} 3.1.3 problem 4

Prove $sup(A \cup B) \geq sup (A)$ and $sup(A\cap B) \leq sup(A)$

\hrule

\begin{proof}

First we show $sup(A \cup B) \geq sup (A)$ by contradiction. We suppose that $sup(A \cup B) < sup(A)$. By definition $sup(A \cup B) \geq x \: \forall x \in A \cup B$. Then since every element in $A$ is also in $A \cup B$ it is true that $sup(A \cup B) \geq x \: \forall x \in A$. Since $sup(A) \geq x \: \forall x \in A$ and $sup(A \cup B) < sup(A)$ then $sup(A)$ is not a least upper bound, a contradiction so $sup(A \cup B) \geq sup(A)$

Next we show $sup(A \cap B) \leq sup(A)$ by contradiction. We suppose that $sup(A \cap B) > sup(A)$. Since $sup(A \cap B)$ is the least upper bound, $\exists x \in A \cap B \; s.t |x - sup(A \cap B)| \leq 1/n \; \forall n$. Clearly everything in $A \cap B$ is also in $A$. Since $sup(A \cap B) > sup(A)$ then $\exists x \in A \cap B > sup(A)$ which is a contradiction since $A \cap B \subseteq A$. So $sup(A \cap B) \leq sup(A)$

\end{proof}








\newpage
\problem{4} 3.1.3 problem 6

Is every subsequence of a subsequence of a sequence also a subsequence of the sequence?
\hrule

Yes. 

\begin{proof}

Let $x_n$ be some sequence, and $x'_n$ be a subsequence. We show that $x''_n$ is also a subsequence of $x_n$. First clearly every element in $x'_n$ is in $x_n$ since $x'_n$ is a subsequence, then it follows that every element of $x''_n$ is an element of $x_n$ by the same reasoning. We need to show that there is a strictly increasing subsequence selection function $f$. There is a subsequence selection function $g$ that selects elements from $x$ to create $x'$, and another subsequence selection function $h$ that selects elements from $x'$ to create $x''$. The subsequence selection function $f = h(g(n))$. $f: \naturals \to \naturals$ since $g: \naturals \to \naturals$ and $h: \naturals \to \naturals$. We show that equation \ref{eq} is strictly increasing 

\begin{equation}
\label{eq}
h(g(n+1)) > h(g(n))
\end{equation}

$g(n+1) > g(n)$ by definition. Let $a = g(n)$ then $g(n+1) \geq a + 1$, so in the worst case we have $g(n+1) = a + 1$. So we substitute this into equation \ref{eq}. $h(a + 1) > h(a)$. This is true because $h$ is a subsequence selection function so is strictly increasing. Thus $f$ which is the composition of $h$ and $g$ ($f = h(g(n))$) must be strictly increasing. So $f$ is a subsequence selection function, and $x''_n$ must be a subsequence of $x_n$ as required.  

\end{proof}





\end{document}