\documentclass[11pt]{article}
\usepackage{../EllioStyle}

\title{Exam 1}
\author{Elliott Pryor}
\date{2 Oct 2020}


\begin{document}
\maketitle

\problem{1}
True or false

\begin{enumerate}
	\item The set of $\rationals$ is countable? 
	
	True
	
	\item Let $A_1, A_2, A_3, ...$ be a sequence of countably many sets, and each $A_i, i = 1,2,3,...$ is countable then their cartesian product $A_1 x \times A_2 \times A_3 \times ...$ i countable.
	
	False
	
	\item A Cauchy sequence of positive rational numbers cannot be equivalent to a Cauchy sequence of negative rational numbers.
	
	False, $x_n = 1/n$ $y_n = -1/n$ both converge to 0 so have the same limit and are thus equivalent.
	
	\item Let $x_1, x_2, ...$ be a convergent sequence of real numbers, then $limsup x_n = liminf x_n$.
	
	True 
	
	\item The union of any number of closed sets is a closed set.
	
	False
	
\end{enumerate}

\newpage
\problem{2}
The statement ``A real number x is a limit-point of a sequence of real numbers $x_1, x_2, . . .$"
can be written explicitly using quantifiers as follows: ``For all $n \in \naturals$, for all $m \in \naturals$, there
exists $j > m$ such that $|x - x_j| < 1/n$". 

Now write the statement ``A real number x is
not a limit-point of a sequence of real numbers $x_1, x_2, ...$" explicitly using quantifiers
\hrule


There exists an $n, m \in \naturals$ such that for all $j > m$ $|x - x_j| > 1/n$

Or equivalently

$$\exists m, n \in \naturals \; s.t. \forall j > m \; |x - x_j| > 1/n$$


\newpage
\problem{3}
Give an example of a set A that is not closed but such that every point of A is a limit point
\hrule

We construct cantor set $C$. Then because $C$ is the cantor set, it is closed and perfect. So it contains all of its limit points and all of its points are limit points.  Then let $A = C \setminus \{0, 1\}$. Then every point in A lies within $(0, 1)$ so it is open. Since we only subtracted two points, every point remaining in $A$ must also be a limit point of $A$ (since $C$ is perfect). But it does not contain all of its limit points (it is missing 0 and 1) so it is not closed. Thus $A$ is not closed and every point in A is a limit point. 



\newpage
\problem{4}
let $x_1, x_2, ...$ be a sequence of real numbers given by
$$x_n = \frac{\cos(\sqrt{n!}\pi)}{4^n}$$
and define
$y_n = x_1 + x_2 + ... + x_n$
prove that the sequence $y_1, y_2, ...$ converges
\hrule

\begin{proof}

We know that $y_n = \sum_{i = 1} ^n \frac{\cos(\sqrt{i!}\pi)}{4^i}$

We want to show that the sequence converges, so we show that it is Cauchy. So we must show that $\forall n \in \naturals \: \exists m \in \naturals \: \forall j,k \geq m \; |y_j - y_k| \leq 1/n$. 
We assume that $j \geq k$, then we have that:
$$|y_j - y_k| = \left| \sum_{i = 1} ^j \frac{\cos(\sqrt{i!}\pi)}{4^i} - \sum_{i = 1} ^k \frac{\cos(\sqrt{i!}\pi)}{4^i} \right| = \left| \sum_{i = k} ^j \frac{\cos(\sqrt{i!}\pi)}{4^i} \right|$$

Then since $\cos(\sqrt{i!}\pi)$ is bounded $[-1, 1]$ 
$\left| \sum_{i = k} ^j \frac{\cos(\sqrt{i!}\pi)}{4^i} \right| \leq \sum_{i = k} ^j \frac{1}{4^i} < \sum_{i = k} ^\infty \frac{1}{4^i}$.
 In order to simplify this further we provide two properties of geometric series $\sum_{n=0}^\infty r^n = \frac{1}{1-r}$ and $\sum_{n=0}^x r^n = \frac{1-r^x}{1-r}$ if $|r| < 1$. Using this we can then solve for an expression for $\sum_{i = k} ^\infty \frac{1}{4^i}$
\begin{align*}
\sum_{i=0}^\infty \frac{1}{4^i} &= \sum_{i=0}^{k-1} \frac{1}{4^i} + \sum_{i=k}^\infty \frac{1}{4^i}\\
\sum_{i=0}^\infty \frac{1}{4^i} - \sum_{i=0}^{k-1} \frac{1}{4^i} &= \sum_{i=k}^\infty \frac{1}{4^i}\\
\frac{4}{3} - \left(\frac{1 - \frac{1}{4^{k-1}}}{1-\frac{1}{4}} \right) &= \sum_{i=k}^\infty \frac{1}{4^i}\\
\frac{1}{3} * \frac{1}{4^k} &= \sum_{i=k}^\infty \frac{1}{4^i}
\end{align*}
Now the smallest $k$ can be is $m$, so $\sum_{i = k} ^\infty \frac{1}{4^i} \leq \sum_{i = m} ^\infty \frac{1}{4^i}$ so we must pick an $m$ such that $\sum_{i=m}^\infty \frac{1}{4^i} = \frac{1}{3} * \frac{1}{4^m} \leq 1/n$. This is true if $\log_4(n/3) \leq m$ for simplicity, we set $m = n$ since $n > \log_4(n/3) \; \forall n$. So in conclusion:

$$|y_j - y_k| \leq \sum_{i = k} ^j \frac{1}{4^i} < \sum_{i = k} ^\infty \frac{1}{4^i} \leq \sum_{i = m} ^\infty \frac{1}{4^i} = \frac{1}{3} * \frac{1}{4^m}$$

Now we have found an $m$ such that $\forall n \in \naturals \; \forall j,k \geq m \: |y_j - y_k| \leq 1/n$. So the sequence is Cauchy. Then it must converge, since all Cauchy sequences are convergent in the reals.

\end{proof}


\newpage
\problem{5}

prove
$$limsup\{x_n + y_n\} \leq limsup\{x_n\} + limsup\{y_n\}$$
if both $limsup\{x_n\}$ and $limsup\{y_n\}$ are finite, and give an example where the equality does not hold.

\hrule

\begin{proof}

 Fist define $z_n = \{x_n + y_n\}$. Let $X = limsup\{x_n\}$ and $Y = limsup\{y_n\}$ and $Z = limsup\{z_n\}$. By the definition of $limsup$ we have that there are infinitely many points in $\{z_n\}$ within a neighborhood of $Z$ (by neighborhood we mean $|z_n - Z| \leq 1/n$). There must also be a convergent subsequence  $z'$ that converges to $Z$. Because there is a subsequence selection function $f: \naturals \to \naturals$: $z'_i = z_j = x_j + y_j$. Then $x_j \leq sup_{a \geq j}\{x_a\}$ and $y_j \leq sup_{a \geq j}\{x_a\}$. We know that the 
 


\begin{align*}
Z &= \lim_{\i \to \infty} z'_i = \lim_{\j \to \infty} z_j = \lim_{\j \to \infty} (x_j + y_j) \\
&\leq \lim_{\j \to \infty} (sup_{a \geq j}\{x_a\} + sup_{a \geq j}\{y_a\})
&= \lim_{\j \to \infty} (sup_{a \geq j}\{x_a\}) + \lim_{\j \to \infty} (sup_{a \geq j}\{y_a\}) \\
&\leq limsup\{x_j\} + limsup\{y_j\}
\end{align*}
 
$$$$

So we have $limsup\{x_n + y_n\} = limsup\{z_n\} = Z \leq limsup\{x_n\} + limsup\{y_n\}$
\end{proof}

The inequality is necessary, if we have $x_n = 0, 1, 0, 1, 0, 1, ...$ and $y_n = 2, -3, 2, -3, 2, -3, ...$ then $z_n = 2, -2, 2, -2, 2, -2, ....$. Clearly $limsup\{z_n\} = 2$ and $limsup\{x_n\} = 1$ and $limsup\{y_n\} = 2$. But $2 \leq 1 + 2$ so the inequality is necessary (ie equality does not hold).





\newpage
\problem{6}
Find the sup, inf, limsup, liminf, and all limit-points of the sequence 
$$x_n = (-1)^n + 1/n + 2cos(\frac{n\pi}{2}, \quad n = 1,2,3...$$
\hrule

We start by listing a few elements to get an idea on the pattern

$x_n = 0, -1/2,-2/3,13/4,-4/5,-5/6,-6/7,25/8,...$

Unless $n \bmod 2 = 0$ the $\cos$ term is zero. So in these cases, the sequence is $-1^n + 1/n$. When $n$ is divisible by $2$ and not by $4$, the cosine term is $-2$, but the $(-1)^n$ term is positive, so $1 - 2 = -1$. But when $n$ is divisible by $4$ both the $(-1)^n$ term and the $\cos$ term are positive. So we get 
$3 + 1/n$ on terms where $n \bmod 4 = 0$ and $-1 + 1/n$ on all the other terms. These two subsequences we denote $y_n = 3 + 1/n$ and $z_n = -1 + 1/n$ where 
$x = z_1, z_2, z_3, y_4, z_5, z_6, z_7, y_8, z_9, .... $


In the first few terms we see that $sup = 13/4$, as this is the first term in the sequence of $y_n$ which is monotone decreasing, so the first term is the largest. Then for the remainder we need to examine the limits of the sequences. For $limsup$ we examine the limit of $y_n$. $\lim_{n \to \infty} y_n = \lim_{n \to \infty} 3  + 1/n = 3$. So the $limsup = 3$. Since $y_n$ is monotone decreasing and bounded, it converges to its one limit point. We show that it converges to 3 $\forall n\in \naturals \; \exists m \; s.t \; \forall j \geq m \; |y_j - 3| \leq 1/n$. We choose $m = n$ then $|y_j - 3| \leq |y_m -3 | = |3 + 1/n -3| = 1/n \leq 1/n$ as required. 

Then to compute the $inf$ and $liminf$ we examine $\lim_{n \to \infty} z_n = \lim_{n \to \infty} -1 + 1/n = -1$. This sequence is also monotone decreasing. So by the same argument as above we can see that the limit point of the sequence is $-1$. This is the $inf$ and $liminf$ of $x_n$.

$sup = 13/4 \quad\quad limsup = 3 \quad\quad inf = -1 \quad\quad liminf = -1$. 


\end{document}