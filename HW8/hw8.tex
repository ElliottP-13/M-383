\documentclass[11pt]{article}
\usepackage{../EllioStyle}

\title{Homework 8}
\author{Elliott Pryor}
\date{28 Oct 2020}
\lhead{Elliott Pryor}

\renewcommand{\sp}{\; \;}

\begin{document}
\maketitle

\problem{1} 4.2.4 Problem 1

If $f$ is monotone increasing on an interval and has a jump discontinuity at $x_0$ in the interior of the domain, show that the jump is bounded above by $f(x_2) - f(x_1)$ for any two points $x_1, x_2$ in the domain surrounding $x_0$: $x_1 < x_0 < x_2$

\hrule

\begin{proof}

Suppose not, suppose that the jump is larger than $f(x_2) - f(x_1)$ for some $x_2, x_1$ in the domain surrounding $x_0$. Let $j$ denote the jump, $f(x_2) - f(x_1) < j$ . Then we evaluate the value of the function, since it is monotone increasing the smallest the gap between $x_1, x_2$ could be is $j$ (ie the function is constant on either side of the jump discontinuity). So $f(x_2) - f(x_1) \geq j$, a contradiction since $f(x_2) - f(x_1) < j$.

\end{proof}




\problem{2} 4.2.4 Problem 3

If the domain of a continuous function is an interval, show that the image is an interval. Give examples where the image is an open interval.
\textit{Hint: Consider the interval with end points inf\{f(D)\} and sup\{f(D)\} where f is the function and D is the domain and use the intermediate Value theorem}

\hrule


\begin{proof}

Let $a = f^-1(\inf\{f(D)\})$ and $b = f^-1(sup\{f(D)\})$. If $a, b$ both exist and are finite then we can construct a closed interval $[a, b]$ which by the intermediate value theorem there must exist some $x' \in [a,b]$ st $f(x') \in [f(a), f(b)]$ which is a closed interval.

Then if $a$ or $b$ does not exist it must asymptotically approach $\inf\{f(D)\}$ or $sup\{f(D)\}$. 
If $\inf\{f(D)\}$ or $sup\{f(D)\}$ are finite, then we can construct closed interval $[a', b']$ where $a' = f^-1(\inf\{f(D)\} + 1/n)$ $b' = f^-1(\sup\{f(D)\} - 1/n)$ for some $n \in \naturals$ which is contained in $D$. 
$\inf\{f(D)\}$ or $sup\{f(D)\}$ are infinite then we can similarly construct a closed interval $[a', b']$ in $D$ where $a' = f^-1(-n)$, $b' = f^-1(n)$. 
Then by intermediate value theorem there is a closed interval $[f(a'), f(b')]$ contained in the image of $f$. 
So the image $f(D) = \cup_{i = 1}^{\infty} [f(a'), f(b')]$ which is an open interval.

Note we assumed $a \leq b$, these can be swapped if $b < a$ and the argument still holds.  


\end{proof}

For example, the function $f(x) = \arctan(x)$ is defined on $\reals$ and its image is $(-1,1)$. Or $f(x) = x$ has domain $\reals$ and image $\reals$.


\problem{3} 4.2.4 Problem 9

If $f$ and $g$ are uniformly continuous, show that $f+g$ is uniformly continuous

\hrule

\begin{proof}

Let $f, g$ be continuous functions defined on a domain $D$. By the definition of uniform continuity $\forall 1/m \sp \exists 1/n_1$ st. $\forall x, x_0 \in D \sp |x - x_0| < 1/n_1 \implies |f(x) - f(x_0)| < 1/2m$ 
and for $g$ $\forall 1/m \sp \exists 1/n_2$ st. $\forall x, x_0 \in D \sp |x - x_0| < 1/n_2 \implies |g(x) - g(x_0)| < 1/2m$. 

Then we want to show $\forall 1/m \sp \exists 1/n$ st. $\forall x, x_0 \in D \sp |x - x_0| < 1/n \implies |f(x) + g(x) - f(x_0) - g(x_0)| < 1/m$. 
We select $n = \min(1/n_1, 1/n_2)$ then: 
$|f(x) + g(x) - f(x_0) - g(x_0)| = |f(x) - f(x_0) + g(x) - g(x_0)| \leq |f(x) - f(x_0)| + |g(x) - g(x_0)| < 1/2m + 1/2m = 1/m$
\end{proof}





\problem{4} 4.2.4 problem 11

If $f$ is a continuous function on a compact set, show that either $f$ has a zero or $f$ is bounded away from zero ($|f(x)| > 1/n$ for all $x$ in domain and some $1/n$). 

\hrule



\begin{proof}

By theorem 4.2.4 we know that the image of $f(D)$ is compact. Then if $0 \in f(D)$ we are done it has a zero. Then if $0 \notin f(D)$ we show it must be bounded away from 0. Since $f(D)$ is compact it contains all its limit points, and $0 \notin f(D)$ so $0$ is not a limit point of $f(D)$. Then by the definition: for any sequence of numbers $x_j$ in $f(D)$ $\exists n \in \naturals$ st. $\exists m \in naturals$ st. $\forall j \geq m \sp |x_j - 0| \geq 1/n$. Since $x_j \in f(D)$, $\exists x \in D \sp st. \sp f(x) = x_j$. So we have for any $x \in D$ $\exists n \in \naturals$ st. $|f(x) - 0| \geq 1/n$ and is thus bounded away from 0.

\end{proof}



\end{document}