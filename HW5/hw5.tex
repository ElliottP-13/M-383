\documentclass[11pt]{article}
\usepackage{../EllioStyle}

\title{Homework 4}
\author{Elliott Pryor}
\date{25 Sept 2020}


\begin{document}
\maketitle



\problem{1} 3.2.3 Problem 1

Let $A$ be an open set. Show that if a finite number of points are removed from $A$ the remaining set is still open. Is the same true if a countable number of points are removed?
\hrule

\begin{proof}

Because $A$ is an open set, we know that $A$ can be represented as the union of disjoint open intervals $\cup^\infty (a_i, b_i)$. 
Furthermore, by the definition of an open set, we know that for any $x \in A$, $x$ is in an open interval that is contained within $A$.
Say that $x \in (a, b)$. Then $x \notin (a, x)\cup(x,b)$. Since every open interval representing $A$ is disjoint, we replace the single interval $(a,b)$ containing $x$ with two more disjoint intervals $(a, x)$ and $(x, b)$. Then $A \setminus \{x\}$ can still be represented by a set of open intervals. So $A$ is open
\end{proof}

No, the same is not true if a countable number of points are removed




\newpage
\problem{2} 3.2.3 Problem 4

Let $A$ be a set and $x$ a number. Show that $x$ is a limit point of $A$ if and only if there exists a sequence $x_1, x_2, ...$ of distinct points in $A$ that converges to $x$.

\hrule

\newpage
\problem{3} 3.2.3 Problem 5

Let $A$ be a closed set, $x$ a point in $A$, and $B$ be the set $A$ with $x$ removed. Under what conditions is $B$ closed. 

\hrule


\end{document}