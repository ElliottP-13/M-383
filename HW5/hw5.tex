\documentclass[11pt]{article}
\usepackage{../EllioStyle}

\title{Homework 5}
\author{Elliott Pryor}
\date{4 Oct 2020}


\begin{document}
\maketitle



\problem{1} 3.2.3 Problem 1

Let $A$ be an open set. Show that if a finite number of points are removed from $A$ the remaining set is still open. Is the same true if a countable number of points are removed?
\hrule

\begin{proof}

Because $A$ is an open set, we know that $A$ can be represented as the union of disjoint open intervals $\cup^\infty (a_i, b_i)$. 
Furthermore, by the definition of an open set, we know that for any $x \in A$, $x$ is in an open interval that is contained within $A$.
Say that $x \in (a, b)$. Then $x \notin (a, x)\cup(x,b)$. Since every open interval representing $A$ is disjoint, we replace the single interval $(a,b)$ containing $x$ with two more disjoint intervals $(a, x)$ and $(x, b)$. Then $A \setminus \{x\}$ can still be represented by a set of open intervals. So $A$ is open
\end{proof}

No, the same is not true if a countable number of points are removed




\newpage
\problem{2} 3.2.3 Problem 4

Let $A$ be a set and $x$ a number. Show that $x$ is a limit point of $A$ if and only if there exists a sequence $x_1, x_2, ...$ of distinct points in $A$ that converges to $x$.

\hrule



\begin{proof}

We prove the forward direction $P \to Q$

We assume that $x$ is a limit point of $A$ and show that there is a sequence of distinct points $x_1, x_2, ...$ in $A$ that converge to $x$. By the definition of a limit point we know that for any $n \in \naturals$ there exists a $y_n \in A$ s.t. $y_n \neq x$ and $|y_n - x| \leq 1/n$. Then by the axiom of Archimedes there must be infinitely many such $y_n$. Since if there was a finite number of $y_n$ then there is some $n_{max}$ and there is some other $n$ satisfying $1/n < 1/n_{max}$. 

We then construct the sequence $y_1, y_2, y_3, ...$ where $y_n$ is the $y_i$ satisfying $\max_{i}\{|y_i - x| < 1/n \}$. Or in other words, we select the point in $A$ that is furthest from $x$ while being in a neighborhood of $1/n$ from $x$. We choose subsequence $y'_i$ such that $y'_i \neq y'_j$ if $i \neq j$. We can do this since there are infinitely many points within $1/n$ of $x$ for any $n$ so we can always find a unique one, and since each $y_n$ it may only be repeated a finite number of times in a row in the original sequence. Thus, $y'_n$ is a sequence of distinct points satsifying $|y'_n - x| < 1/n$ $\forall n$. Thus the sequence $y'_n$ converges to $x$. So if $x$ is a limit point of $A$ there exists a sequence of distinct points in $A$ that converge to $x$.


We prove the reverse direction $P \gets Q$

We assume that there is a sequence of points $x_1, x_2, ....$ of distinct points in $A$ that converge to $x$. We know that since $x_1, x_2, ...$ converges to $x$ then $\forall n \in \naturals \; \exists m \; s.t. \forall j \geq m \; |x_j - x| < 1/n$. 
Since the sequence is a sequence of distinct point we know that $x_j \neq x$ for any $j$. Then we have for any $n \in \naturals$ there exists a $x_j \in A$ s.t. $x_j \neq x$ and $|x_j - x| < 1/n$.
Which is the definition of a limit point of the sequence, so $x$ is a limit point of $A$

\end{proof}


\newpage
\problem{3} 3.2.3 Problem 5

Let $A$ be a closed set, $x$ a point in $A$, and $B$ be the set $A$ with $x$ removed. Under what conditions is $B$ closed. 

\hrule

A set is closed if it contains all of its limit points. So if $B$ is closed $x$ must not be a limit point of $A$. Since removing a single element from a set will not change its limit points. 


\end{document}